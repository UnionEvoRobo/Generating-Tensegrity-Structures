 \documentclass[11pt]{article}
%\usepackage{epsf}
%\usepackage{psfig}
\usepackage{graphicx}
\special{papersize=8.5in,11in}		% Hey dvips, this is America!
\setlength{\oddsidemargin}{0in}		% Actual left margin - 1 inch. 
\setlength{\evensidemargin}{0in}	% Ditto.                       
\setlength{\textwidth}{6.5in}		% Line length.                 
\setlength{\marginparwidth}{0.5in}	% Width of margin remarks.
\setlength{\topmargin}{0in}		% Actual top margin - 1 inch.
\setlength{\headheight}{0in}		% No page headers.
\setlength{\headsep}{0in}		% Space between head and body. 
\setlength{\textheight}{9.1in}		% Body height (incl. footnotes).
\setlength{\footskip}{0.3in}		% Space between footnotes.
\setlength{\baselineskip}{16pt minus 1pt}	% Space betwen lines.
\setlength{\parskip}{3pt plus1pt minus.5pt}	% Space between paragraphs.
\begin{document}
\ \\
\newcommand{\mrk}{Keijzer}
\newcommand{\rab}{``{\em Representation and Behavior}''}
\begin{center}
\Large Evolving Robustly Buildable Assembly Procedures\\

John Rieffel and Jordan Pollack
\end{center}

\section{Introduction}
\section{Theory and Background}
\section{Experiments}
\section{Conclusion}
Natural embryogenic processes unfold in a world which is intrinsically
stochastic and noisy.  When genotypes are given multiple opportunities
to embryogenically produce a phenotype,  it would seem that a noisy
embryogenic environment would cause significant variation on the set of
phenotypes that result.  Yet, the world is full of genotypes and
embryogenic systems (genotype plus embryogeny) that reliably achieve
extremly complex designs.  

In the context Evolution of Design, approaches using embryogenic
models, in which the genome is seen as an assembly plan for the
construction of a physical desigh,  have successfully achieved
complex design (cf. Hornby, Bongard et al), but are blind to the
prospect of noisy assembly, and hence brittle in its presence.  The
problem of creating assembly procedures that are robust to noisy
assembly is thus presented.

If noise during assembly imposes a distribution on the structures that 
result from a single assembly plan, the goal is to develop assembly
plans that are robust to this noise.  {\em Robustness} in an assembly
plan is the property of being able to consistently produce
satisfactory results over the course of multiple nondeterministic
constructions. Robustness {\em does not} require that all of an assembly
plan's satisfactory results be similar in some external subjective
sense, i.e. similar in structure, merely that they are comparably
fit.

Adding a simple extension to evolutionary approach of earlier efforts
(the "algorithm" I described in the earlier post, in which each assembly
plan is built into 100 results, and these results are then selected for
over the standard objectives), I'd like to show, can result in assembly
plans that are rubost in the face of noisy assembly without any explicit
or external judgement (i.e. the "yield" metric I had been using before)
of rubusticity.

Introducing noise into the embrygenic model allows us to explore
several important questions that lie at the intersection of
evolutionary biology and evolution-of-design:

- genome-as-blueprint vs. genome + environment (process) as
(self-correcting) program

- emergence of canalization

- emergence of modularity as stabilizing/robusticizing agent in
ontogenic development.

- development of embrygenic ``checkpoints'' as ontogenic temporal
module

- build-time vs evol-time sensitivity (richard)

-''inherently precise stochastic process''
 
Shiva Sez:

 The "genome-as-blueprint" is done away with completely. All that 
   matters is that the participation of the genome in the physics of 
   construction results in "fit" outcomes. When these "fit" outcomes
   are structurally identical only then is the genome seemingly acting
   as if it were a blueprint. So the "genome-as-blueprint" is really 
   just a special case of a general one where the genome results in
   functional offspring reliably. And the other extreme of this continuum 
   is what your algorithm does at present, where it accepts all outcomes
   independent of similarity or the lack of it.

 Selection acts on individual phenotypes rather than on the NOMINAL or 
   REPRESENTATIVE individual computed using heuristic statistical 
   measures on populations of individuals from the same genome. 


\bibliography{mybib}
\bibliographystyle{plain}
\end{document}
